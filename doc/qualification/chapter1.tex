%!TEX root = ./qualification.tex

\newpage
\section{Capítulo 1 - Introdução}
Em um ambiente manufatureiro, a programação da produção consiste em alocar recursos disponíveis (pessoas, máquinas, veículos, etc) requeridos para executar tarefas de produção de um determinado conjunto de produtos. Identificar a ordem de produção e o tempo de execução dessas tarefas é buscar uma ótima programação - aquela cujos esforços são menores e que, consequentemente, aumentam a competitividade da empresa no mercado.

\subsubsection{Programação Reativa}
No cotidiano do ambiente de produção, é comum acontecerem a inclusão de novos pedidos, cancelamento de pedidos que já estão com tarefas em andamento, quebra de máquinas, ausência de recursos, etc. A programação da produção deve ser capaz de reprogramar-se, para responder efetiva e rapidamente a quaisquer eventos, afim de minimizar ou até mesmo extinguir o tempo ocioso. Isso é chamado de programação reativa da produção ou reprogramação da produção \cite{SUN}.

    \subsubsection{Makespan}
    Lorem ipsum
        
    \subsubsection{Tamanho do problema}
    Lorem ipsum

\subsection{Justificativa}
Lorem ipsum...

\subsection{Objetivos}
Lorem ipsum...

\subsection{Metodologia}
Lorem ipsum...