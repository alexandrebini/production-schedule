%!TEX root = ./qualification.tex

\newpage
\section{Capítulo 2 - Algorítmos Genéticos}
\label{sec:capitulo_2}
Lorem Ipsum

\subsection{Fundamentos}
\label{sub:fundamentos}
A programação genética se baseia no princípio de reprodução e sobrevivência dos mais aptos de Charles Darwin, simulamando o processo biológico de evolução.

Existem diversas áreas de aplicação, podendo-se destacar: Otimização (propósito deste trabalho), Programação Automática, Aprendizado de Máquina, Modelos econômicos/sociais/ecológicos, Interações entre evolução e aprendizado, etc.

\subsection{Função de aptidão}
\label{sub:funcao_de_aptidao}
Lorem ipsum...

\subsection{Métodos de seleção}
\label{sub:metodos_de_selecao}
Lorem ipsum...

    \subsubsection{Roleta}
    \label{ssub:roleta}
    Lorem ipsum

    \subsubsection{Torneio}
    \label{ssub:torneio}
    Lorem ipsum

\subsection{Mecanismos de reprodução}
\label{sub:mecanismos_de_reproducao}
Lorem ipsum...

    \subsubsection{Mutação}
    \label{ssub:mutacao}
    Lorem ipsum

    \subsubsection{Cruzamento}
    \label{ssub:cruzamento}
    Lorem ipsum

\subsection{Algorítmos Genéticos Adaptativos (AGA)}
\label{sub:aga}
Lorem ipsum...
